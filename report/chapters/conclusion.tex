\section{Выводы}
Предложенный подход к оптимизации подбора гиперпараметров нейронных сетей позволяет значительно ускорить этот длительный процесс.
Однако малое количество данных оставляет много вопросов, и, несомненно, одним из возможных дальнейших направлений для исследования может стать
проведение эксперимента с значительно большим числом испытаний.
Также интересна устойчивость метрики: насколько много ситуаций, когда мы отсечём лучшее испытание?
Кроме того, насколько вероятна ситуация, когда отсечено будет не просто лучшее испытание, но испытание, которое \textit{заметно} лучше оставшихся?
Наконец, стоит исследовать применимость других метрик схожести сетей, кроме CKA, и проверить предложенный метод в других областях машинного обучения.