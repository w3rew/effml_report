\section{Обзор}
Оптимизация гиперпараметров --- это обширная тема, и ей посвящено большое количество обзоров, например~\cite{yu2020hyper}.

Оценка похожести нейронных сетей осуществляется обычно исходя из выходов нейросети (не весов).
Если есть выходы $X \in \mathbb{R}^{m\times n}$ и $Y \in \mathbb{R}^{k\times t}$ разных слоёв одной и той же, либо разных моделей, то для определения их схожести есть следующие методы:
\begin{itemize}
    \item Линейная регрессия между $X$ и $Y$ --- коэффициентов является $R^2$
    \item CCA~\cite{ramsay1984matrix}. Находятся базисы, после проекции $X$ и $Y$ на которые корреляция максимизируется.
    \item SVCCA~\cite{golub1995canonical}. То же самое, но после SVD-разложения.
    \item Projection-Weighted CCA~\cite{morcos2018insights}
    \item Mutual information
    \item CKA~\cite{kornblith2019similarity}
\end{itemize}

CKA является широко принятым методом, поэтому я буду использовать его, в его линейном варианте.
За дальнейшими деталями отсылаю к оригинальной работе.